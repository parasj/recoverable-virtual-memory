\documentclass{report} % A4 paper and 11pt font size [paper=letter,fontsize=12pt]

\usepackage[T1]{fontenc} % Use 8-bit encoding that has 256 glyphs
\usepackage{fourier} % Use the Adobe Utopia font for the document - comment this line to return to the LaTeX default
\usepackage{graphicx} % For pictures
\usepackage{booktabs} % For tables
\usepackage{tabularx} % For tables
\usepackage[english]{babel} % English language/hyphenation
\usepackage{amsmath,amsfonts,amsthm} % Math packages
\usepackage{listings}
\usepackage[margin=1.5in]{geometry}

\usepackage{lipsum} % Used for inserting dummy 'Lorem ipsum' text into the template

\usepackage{sectsty} % Allows customizing section commands
\allsectionsfont{\centering \normalfont\scshape} % Make all sections centered, the default font and small caps

\usepackage{fancyhdr} % Custom headers and footers
\pagestyle{fancyplain} % Makes all pages in the document conform to the custom headers and footers
\fancyhead{} % No page header - if you want one, create it in the same way as the footers below
\fancyfoot[L]{} % Empty left footer
\fancyfoot[C]{} % Empty center footer
\fancyfoot[R]{\thepage} % Page numbering for right footer
\renewcommand{\headrulewidth}{0pt} % Remove header underlines
\renewcommand{\footrulewidth}{0pt} % Remove footer underlines
\setlength{\headheight}{13.6pt} % Customize the height of the header

\numberwithin{equation}{section} % Number equations within sections (i.e. 1.1, 1.2, 2.1, 2.2 instead of 1, 2, 3, 4)
\numberwithin{figure}{section} % Number figures within sections (i.e. 1.1, 1.2, 2.1, 2.2 instead of 1, 2, 3, 4)
\numberwithin{table}{section} % Number tables within sections (i.e. 1.1, 1.2, 2.1, 2.2 instead of 1, 2, 3, 4)

% \setlength\parindent{0pt} % Removes all indentation from paragraphs - comment this line for an assignment with lots of text

\renewcommand{\thesection}{\arabic{section}}

%----------------------------------------------------------------------------------------
%	TITLE SECTION
%----------------------------------------------------------------------------------------

\newcommand{\horrule}[1]{\rule{\linewidth}{#1}} % Create horizontal rule command with 1 argument of height

\title{	
\normalfont \normalsize 
\textsc{Georgia Institute of Technology} \\ [25pt] % Your university, school and/or department name(s)
\horrule{0.5pt} \\[0.4cm] % Thin top horizontal rule
\Huge Recoverable Virtual Memory \\ \huge CS 6210 Project 3 Report\\ % The assignment title
\horrule{2pt} \\[0.5cm] % Thick bottom horizontal rule
}

\author{Paras Jain, Manas George}
\date{}

\begin{document}

\maketitle % Print the title

\section{Project goal}
The aim of this project is to implement a recoverable virtual memory system. This system allows users to manage peristent memory using a transaction-based API. This mechanism allows clients to create applications with persistant and consistent data structures (e.g. a database).

\section{API}

\begin{enumerate}
  \item \texttt{rvm\_init}: initializes the log files and directory for the VM system
  \item \texttt{rvm\_map}: maps a segment file on disk to area in memory
  \item \texttt{rvm\_unmap}: unmaps a segment from memory
  \item \texttt{rvm\_destroy}: cleans up the backing store for a segment
  \item \texttt{rvm\_begin\_trans}: creates a transaction
  \item \texttt{rvm\_about\_to\_modify}: notifies library about incipient modification to memory segments
  \item \texttt{rvm\_commit\_trans}: commits the transaction to disk
  \item \texttt{rvm\_abort\_trans}: undo all modifications during a trasaction
  \item \texttt{rvm\_truncate\_log}: shrink the log file and write log data
  \item \texttt{rvm\_verbose}: control log verbosity
\end{enumerate}


\section{RVM design and implementation}
Upon calling \texttt{rvm\_init}, a directory is created if it does not exist along with a log file. Relevant segement list data structures are created as well. When a mapping is created with \texttt{rvm\_map}, a segment file is created if it does not exist. This file stores the memory segment in raw binary format. When creating a transaction, a new transaction ID is created. This transaction ID is checked to prevent multiple transactions from editing the same segment. Calling \texttt{rvm\_about\_to\_modify} creates a range record. Upon commit, the ranges are actually written out to the log file. Finally, in truncate, we read each item from the log file and then apply the changes and write back the data to disk.

This architecture keeps the code simple, which helps ensure that our implementation is more likely to be correct. Give


\subsection{RVM data structures}

\subsubsection{Segments}
Segments are stored in a global segment list. This is a linked list based on the \texttt{segment\_t} struct. In this segment, we store segment size, name and address.

\vbox{%
\begin{lstlisting}[language=C]
typedef struct segment_t {
  char* seg_name;
  int seg_size;
  void* seg_addr;
  trans_t trans_id;

  LIST_ENTRY(segment_t) next_seg;
} segment_t;

struct segment_list segments;
\end{lstlisting}}


\subsubsection{Ranges}
Ranges represent operations on a range of memory. These transactions are used as atomic blocks which allow for undo functionality. These ranges are allocated when a transaction is created (for the undo log) and whenver a client is about to modify something.

\vbox{%
\begin{lstlisting}[language=C]
typedef struct range_t {
  int tid;
  int offset;
  int size;
  int is_undo;
  short is_backed;
  int namesize;
  void* segbase;
  void* data;
  char *segname;

  LIST_ENTRY(range_t) next_range;
} range_t;
\end{lstlisting}}


\subsubsection{Log files}
The commit log file is stored in \texttt{/.rvm.log}. This file stores a list of entries corresponding to transaction commit records. When \texttt{rvm\_commit\_trans} is called, relevant regions are written to the logfile. We copy the updated memory region to the \texttt{range\_t} data structure and then write to the file.

The file is in binary format for speed of processing. An entry in the log is a binary string as follows: \texttt{<range\_t entry> <segname> <data>}. This saves the metadata for the entry with the associated data that the metadata points to. This is sufficient to apply the transactions against the persistent backing file.


\subsubsection{Undo records}
\texttt{rvm\_about\_to\_modify} notfies the library to save an undo record. This saves a \texttt{range\_t} entry into a linked list in memory. This allows cancelling a transaction and restoring previous state, simply by copying back the undo record data.


\section{Analysis of design tradeoffs}

Our implementation is simple which keeps the code relatively bug-free. All library code lives in \texttt{rvm.c} in several interface methods, outlined in the API section. Total implementation is on the order of about 350 LOC. This has positive benefits for the library, notably security and verifiability. However, this led to some potential underutilized performance optimizations.

For example, segments are stored in a single global segment list. Operations on this segment list are on the order of $O(n)$ as each access requires traversing the list to find the relevant segment metadata struct. Using a list keeps implementation simpler and more secure. The list uses \texttt{sys/queue.h} which is a very well tested and robust list implementation using compiler macros. However, we could use a hashmap datastructure (e.g. Google sparse hash) which would have an constant access overhead. Looking though the examples, we decided that most implementations did not use a large enough number of segments to make such an optimization worthwhile. Given that our RVM library is intended to be linked directly into the client application as compared to a system-wide service, segment lists would be scoped to the context of a single application and therefore should not be an issue for the kinds of clients using our library. We therefore avoid needing to link another third-party library into our code.

There are several optimization that could further improve performance of the application. The log currently is read and written to using \texttt{fread}. This interface is performant but incurs the cost of a system call for every \texttt{range\_t} item in the log. A faster approach would be to \texttt{mmap} the transaction log which would batch reads to a system-page granularity. This would also have the added benefit of potentially allowing for page-sharing between clients in the future. It is also likely recently written entries in the transaction log will be read shortly after - \texttt{mmap} would benefit from this as those pages would likely still be cached in memory.

It would also make sense to automatically truncate the log if the logfile grows beyond a certain threshold. We opted to allow the user control of this functionality.

Actually truncating the transaction log is potentially slow as we create a new commit log file and apply each transaction to each mapped segment. For any unmapped segments, we copy those ranges to the new log file. Finally, we move the new file to replace the old one. This is somewhat inefficient but is potentially safer. However, when we write to disk, we do not wait to \texttt{fsync} the written bytes to disk. This could mean if the system were to lose power, this data would still be in a filesystem buffer.

\end{document}